% !TeX program = pdfLaTeX
\documentclass[12pt]{article}
\usepackage{amsmath}
\usepackage{graphicx,psfrag,epsf}
\usepackage{enumerate}
\usepackage{natbib}
\usepackage{textcomp}
\usepackage[hyphens]{url} % not crucial - just used below for the URL
\usepackage{hyperref}

%\pdfminorversion=4
% NOTE: To produce blinded version, replace "0" with "1" below.
\newcommand{\blind}{0}

% DON'T change margins - should be 1 inch all around.
\addtolength{\oddsidemargin}{-.5in}%
\addtolength{\evensidemargin}{-.5in}%
\addtolength{\textwidth}{1in}%
\addtolength{\textheight}{1.3in}%
\addtolength{\topmargin}{-.8in}%

%% load any required packages here



% tightlist command for lists without linebreak
\providecommand{\tightlist}{%
  \setlength{\itemsep}{0pt}\setlength{\parskip}{0pt}}




\begin{document}


\def\spacingset#1{\renewcommand{\baselinestretch}%
{#1}\small\normalsize} \spacingset{1}


%%%%%%%%%%%%%%%%%%%%%%%%%%%%%%%%%%%%%%%%%%%%%%%%%%%%%%%%%%%%%%%%%%%%%%%%%%%%%%

\if0\blind
{
  \title{\bf TITLE TBD}

  \author{
        Lillian Fok \thanks{The authors gratefully acknowledge Professor
Scott LaCombe, Smith College.} \\
    Department of Statical \& Data Sciences, Smith College\\
     and \\     Michele Sezgin \\
    Department of Statical \& Data Sciences, Smith College\\
      }
  \maketitle
} \fi

\if1\blind
{
  \bigskip
  \bigskip
  \bigskip
  \begin{center}
    {\LARGE\bf TITLE TBD}
  \end{center}
  \medskip
} \fi

\bigskip
\begin{abstract}
Twitter has become an increasingly used social networking resource for
politicians, social movements, and citizens. Politicians often use
twitter to communicate with the public, campaign, advocate for
themselves and their policies, and send out calls to action. Twitter has
also provided a convenient and public way for the public and
organizations to call on elected officials to take action on relevant
issues facing the public. We investigate whether state governors twitter
habits regarding the Black Lives Matter movement correlates with passage
of police reform laws in the state policy diffusion network.
\end{abstract}

\noindent%
{\it Keywords:} 3 to 6 keywords, that do not appear in the title
\vfill

\newpage
\spacingset{1.45} % DON'T change the spacing!

\hypertarget{introduction}{%
\section{Introduction}\label{introduction}}

\(\hspace{.5cm}\) Twitter users often observe elected officials
communicating their stance on political issues or making promises of
policy change via Twitter. Politicians and their teams invest
significant effort into curating a tweets and communicating with the
public via online social networking sites. When evaluating the political
performance of politicians, we expect that they ``practice what they
preach'' and at least make good-faith attempts to pass legislation
regarding issues they have publicly tweeted their support for. However,
it is difficult to know whether tweeting about an issue predicts passage
of legislation without actively seeking out this information. In the
context of the Black Lives Matter movement, politicians were facing
unprecedented societal pressure to crack down on police violence and
enact substantial police reform laws. Though a controversial issue that
saw many bills die in committee in Congress, the states had more power
to pass police reform legislation if pursued. This provides the
motivation of our work investigating whether tweeting about the Black
Lives Matter movement correlates with police reform policy adoption.

\hypertarget{literature-review}{%
\section{Literature Review}\label{literature-review}}

\hypertarget{politician-use-of-twitter}{%
\subsection{Politician use of Twitter}\label{politician-use-of-twitter}}

\(\hspace{.5cm}\) Social media has become an increasingly pervasive
presence in our daily lives and its impact has been far-reaching.
Consequently, social media has been increasingly used in a political
context. There are several studies regarding the use of social media in
general and in a political context. Many studies have focused on the way
politicians use social media, finding that common motivations are a
desire to grow their support base (sometimes by inciting an extreme
reaction), to disseminate information, and to network with other
politicians. \citet{Hemsley} finds that in the context of the U.S. 2014
gubernatorial election, candidates for governor used Twitter to advocate
for themselves, their policies, what they believe in, and
calls-to-action in addition to attacks on their opponents and their
policies. Further, Hemsley finds that the the most popular tweets that
reach the widest audience are call-to-action and attack tweets.
Therefore political candidates wishing to broaden their support base may
disproportionately send out attack or call-to-action tweets. In the
Korean political context, \citet{Park} find that Tweets from leading
government officials are more likely to increase citizen perceptions of
credibility in a government Twitter feeds and government overall. This
use of Twitter as a mechanism for politicians to increase trust in
government also appears in the U.S., where \citet{Song} find that
government use of social media websites increases perceived government
transparency and citizens' trust in government. Notably, Song and Lee's
work used 2009 survey data, and may not be consistent with current
public perception of government Twitter feeds, and Park et al.'s work
may not be applicable in the U.S. political context. Former U.S.
president Donald Trump, the highest elected official in the U.S. from
2016-2020, frequently made false or misleading claims on Twitter, and
these false or misleading tweets were among his most popular (Rattner,
2021).

\(\hspace{.5cm}\) The way politicians are using Twitter can also be
broken down by party, gender, and incumbency, among other factors.
\citet{Evans} offer a more granular analysis of how politicians are
using Twitter in their study of candidates for the 2012 U.S. House. They
find that women tweet more overall and are more likely to criticize
their opponents. It may be true that female candidates have to work
harder to grow their support base, as they are underrepresented in many
areas of U.S. government, and may face gender-specific roadblocks to
getting elected (Bos et al., 2018). Paired with Hemsley's findings, it
is possible that female political candidates tweet more
opponent-criticizing or ``attack'' messages in an attempt to produce
more viral tweets and grow their support base. However, our study
focuses on state governor Tweeting during the height of the Black Lives
Matter movement, and thus the findings of Evans et al.~may not correlate
with trends observed in our data. Additionally, our data was not
collected during an election cycle, and thus opponent-attacking and
campaigning tweets are likely not as prevalent in the dataset. Pivoting
to studying politician use of Twitter specific to the BLM movement,
\citet{Panda} find that in a study of Tweets from 520 U.S. Congress
members that Democrats are more likely to tweet about the movement in
general and express their concern for police brutality, while
Republicans are less likely to tweet about the movement overall and more
likely to express concern about perceived protest violence associated
with the movement. Because of these findings, we hypothesize that
Democratic governors will be more likely to tweet about BLM and police
reform overall.

\(\text{Hypothesis}_1: \text{Democratic governors will be more likely to tweet about BLM and police} \\ \text{reform overall.}\)

\hypertarget{black-lives-matter-movement-and-twitter}{%
\subsection{Black Lives Matter movement and
Twitter}\label{black-lives-matter-movement-and-twitter}}

\(\hspace{.5cm}\) The Black Lives Matter (BLM) movement, which peaked in
support following the murder of George Floyd, and declined in support
from June 2020 - August 2020, has had a steady support level since
(Horowitz). Twitter and social media websites played critical roles in
helping the movement build momentum during this time period.
\citet{Mundt} note that Twitter was critical in helping the BLM movement
expand and strengthen its internal ties by decreasing roadblocks to
organizing and amplifying narratives, among other mechanisms.
\citet{Freelon} find that BLM, categorized as a powerful Twitter social
movement, was correlated with increased mainstream news coverage of
issues relevant to the movement, such as police brutality, which is the
strongest attention driver for political elites. Americans look to their
elected officials for leadership and support regarding contentious and
relevant public issues, and this is often portrayed by citizens and
organizations communicating with politicians via Twitter and calling on
them to take action on important issues. For example, \citet{CTEQI}
recently tweeted that Congress should revive efforts to pass the Justice
in Policing Act, a call to action that is not uncommon among social
movements active on Twitter. Additionally, the \citet{BLM} tagged Nancy
Pelosi in a tweet responding to her commentary on the death of George
Floyd, showcasing that Twitter is a direct communication line to public
officials that is more convenient and public than traditional methods of
citizen-politician communication such as letter-writing or emailing.

\(\hspace{.5cm}\) Citizens rely on elected officials to implement policy
and make change when the public is faced with injustices. When
politicians tweet support for certain policies or ideas, this is
important, but besides declarations of signing bills into law or
cosponsoring legislation, how does the public know if this support holds
up when it comes time to vote? For example, \citet{Inslee}, governor of
the state of Washington, tweeted during the height of the BLM movement's
popularity that Washington's state government is ``going to take a hard
look at how we manage independent investigations of police use of force
in WA''. However, it is not know empirically whether this show of
support resulted in legislative action being taken unless citizens seek
it out or he publicly communicates the passage of a law, as
Massachusetts governor \citet{Baker} and California governor
\citet{Newsom} did when they passed comprehensive police reform
legislation in 2020 and new standards for police use of force in 2019,
respectively.

\hypertarget{state-policy-adoption}{%
\subsection{State Policy Adoption}\label{state-policy-adoption}}

\(\hspace{.5cm}\) State policy reform is a complex process that cannot
be captured by the actions of an individual state governor alone. There
has been much scholarship in the field of policy diffusion and
investigation of the factors and underlying network that cause states to
adopt certain policies. In their study of the correlation between
perceived state similarity and similar policy adoption, \citet{Bricker}
find that factors like perceived state similarity, legislative
professionalism, population size differences, and different partisan
control of state legislatures can significantly impact whether a state
will adopt a policy similar to a policy previously adopted by another
state. \citet{Desmarais} model a latent underlying state policy
diffusion network that correlates with media state policy emulation
stories, finding that California tops the list, especially in more
recent years (2005-2009), as a leading policy innovator. In Gavin
Newsom's tweet regarding California's passage of new standards for
police use of force in 2019, he says the passage ``{[}makes
California{]} a model for the rest of the nation'', highlighting that he
likely expects the rest of the states to observe the efficacy of this
legislation and adopt it if successful. This showcases an important
factor in policy diffusion, learning (proposed by \citet{Shipan}), in
which states may observe policy efficacy in other states before deciding
to implement it themselves. Given the results of Desmarais et al.~and
Governor Newsom's statement, it is possible that we will find California
to be a policy innovator in our network. Overall, we know that there is
non-independence among states regarding choice of policy adoption, which
is something we control for in our analysis by treating state policy
adoption as a network object.

\(\text{Hypothesis}_2: \text{California will be a policy innovator in our network, adopting police reform} \\ \text{legislation early on in the diffusion process.}\)

\hypertarget{the-relationship-between-twitter-and-police-reform-policy-adoption}{%
\subsection{The Relationship Between Twitter and Police-Reform Policy
Adoption}\label{the-relationship-between-twitter-and-police-reform-policy-adoption}}

\(\hspace{.5cm}\) From the selected studies and tweets, we've found that
politicians use Twitter for increasing credibility and transparency,
building a support base, and communicating their support for policies
and ideas, and that Twitter has immense power in helping social
movements expand their reach and influence, and communicate with
political elites. However, no studies have been found that investigate
whether promises made, stances declared, or movements talked about by
politicians via Twitter correlate with how they implement policies. We
investigate whether tweeting about the Black Lives Matter movement
during its peak correlates with police reform policy adoption in U.S.
states. We hypothesize that a state governor tweeting about BLM will
positively correlate with that state adopting some level of police
reform policy.

\(\text{Hypothesis}_3: \text{A state governor tweeting about BLM will positively correlate with that} \\ \text{state adopting some level of police reform policy.}\)

\bibliographystyle{agsm}
\bibliography{bibliography.bib}


\end{document}
